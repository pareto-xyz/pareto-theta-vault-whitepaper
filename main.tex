\documentclass[hidelinks, 12pt]{article}

\usepackage{subcaption}
\usepackage{caption}
\usepackage{fullpage}
\usepackage[utf8]{inputenc}
\usepackage{appendix}
\usepackage{graphicx}
\usepackage{amsmath}
\usepackage{amssymb}
\usepackage{amsthm, amsfonts}
\usepackage{booktabs}
\usepackage{listings}
\usepackage{hyperref}
\usepackage{float}

\parskip = \baselineskip
\setlength{\parindent}{0pt}

\title{Pareto Replicating Theta Vaults}

\author{Mike Wu, Will McTighe \\ \small\texttt{\{mike, will\}@paretolabs.xyz}}

\date{February 2022}

\begin{document}

\maketitle

\tableofcontents

\begin{abstract}
Theta Vaults are a simple on-chain structured product that runs an automated strategy to sell rolling weekly out-the-money covered calls in which the premium is returned to vault participants as yield.
Traditionally, Theta Vaults face two main challenges: first, they require oracles which can introduce new vulnerabilities; second, the yield is capped by the efficiency of a matching process to find options buyers.
In this paper, we present the design for Theta Vaults built on top of replicating market makers that approximate a Black-Scholes covered call.
Beyond covered calls, an ecosystem of efficient and secure options vaults can be built by a similar replication.
\end{abstract}

\section{Preface}

This paper is a technical introduction to Pareto Replicating Theta Vaults, a redesign of the yield instrument popularized by Ribbon Finance. In the following, we describe how to construct Theta Vaults on top of replicating market makers (RMMs), in which selling a covered call is equivalent to being a liquidity provider. Our primary hypothesis is that the ability of RMMs to replicate a wide array of structured products can be used to design an options ecosystem with token compositionality as a first-class citizen. With regards to Theta Vaults in particular, RMMs offer a oracle-free and auction-free solution that users may find attractive. In the following sections, we first provide background information then present the mechanism underlying the Replicating Theta Vault (RTV).

\paragraph{[6/15/22]} The alpha product for Pareto RTVs is currently under development. Please reach out to \texttt{mike@paretolabs.xyz} if you have any questions.

\section{Background}

We provide an overview for concepts required to understand RTVs. We opt to provide a high-level explanation and leave a more mathematical treatment to the cited work.

\subsection{Constant Function Market Makers}

Constant function market makers (CFMMs) are a family of automated market makers (AMMs) used as token swap mechanisms on public blockchains. Liquidity providers (LPs) lend a token pair (e.g. Ethereum and USDC) to a supply of reserves managed by a smart contract. Users wishing to swap tokens can call the smart contract, paying a small fee that is rewarded to the LPs. However, the contract is programmed to ensure that a function of the token reserves remains constant, hence the name ``constant function''. This function is called the trading function or the invariant.

A common invariant is the product of the two reserves, used by Uniswap \cite{angeris2019analysis,adams2021uniswap} and Balancer \cite{martinelli2019non} among others. To introduce the notation we will carry throughout this paper, let $x, y \in \mathbf{R}_+$ be reserves for a token pair and $\psi: \mathbf{R} \times \mathbf{R} \rightarrow \mathbf{R}$ be an invariant. For a fixed level of liquidity, we have $\psi(x, y) = k$ for a constant $k \in \mathbf{R}$. A trade of $\Delta x$ for $\Delta y$ is allowed if and only if $\psi(x - \Delta x, y + \Delta y) = k$ for the same value $k$. For any AMM, the relative price is a function of the reserves. In this case, the (marginal) price of token $y$ in token $x$ is simply $\left(\frac{y}{x}\right)$. Other invariants include the arithmetic sum ($\psi = x + y$) along with more complex ``mixtures'' of sum and product invariants such as Yieldspace \cite{niemerg2020yieldspace} or Curve \cite{egorov2021automatic}.

CFMMs and the broader class of AMMs, stand as an alternative to more traditional orderbook markets that require matching between buyers and sellers at various prices. CFMMs require no middlemen, expose transparent prices, have efficiency benefits as trades execute immediately, are oracle-free and hence less vulnerable to malicious actors, and are more robust under black-swan events. Due to these advantages, CFMMs have grown to dominant the swap market, with Uniswap being the largest DEX and fourth largest exchange.
Much more can be said about CFMMs, including impermanent loss and the structure of trading fees, but for this discussion, we recommend reading \cite{angeris2019analysis}.

\subsection{Replicating Market Makers}

TODO.

\subsection{Options Overview}

TODO.

\subsection{Replicating Black-Scholes Covered Calls}

TODO.

\section{Replicating Theta Vaults}

TODO.

\subsection{Mechanism}

TODO.

\subsection{Options Strategy}

TODO.

\subsection{Fee Structure}

TODO.


\section{Limitations and Future Work}

TODO.

\section{Conclusion}

TODO.

\bibliographystyle{unsrt}
\bibliography{main}

\end{document}
